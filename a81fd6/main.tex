% !TeX encoding = UTF-8
% !TeX program = xelatex
% !TeX spellcheck = en_US

\documentclass[a4paper]{ltxdoc}
\usepackage{amsmath}
\usepackage[UTF8]{ctex}
\usepackage{unicode-math}
\usepackage{caption}
\usepackage{booktabs}
\usepackage{xcolor}
\usepackage{array}
\usepackage{listings}
\usepackage[perpage]{footmisc}
\usepackage{hypdoc}
\usepackage{geometry}
\usepackage{endnotes}
% \usepackage[multiple]{endnotes}
\usepackage{multicol}
\usepackage{blindtext}
\geometry{a4paper, scale=0.85}
\newenvironment{Figure}
  {\par\medskip\noindent\minipage{\linewidth}}
  {\endminipage\par\medskip}

\title{实验报告\\声速的测量}
\author{少年班学院\\马天开 PB21000030(1号)}
\date{\today}

\begin{document}
\begin{multicols}{2}
    \maketitle
    \section{实验目的}
    在给定的器材下,进行简单的实验设计和实验基本方法的训练,学会实验器材的使用、测量方法和应用误差均分原理;分析误差的来源,提出修正和估算的方法
    \section{实验器材}
    低频信号发生器、测量仪、示波器、黄铜棒、有机玻璃棒、游标卡尺、温度计。

    发生器频率分度:$\Delta f =1Hz$,游标卡尺精度:$\Delta \approx 10^{-3} cm$
    \section{实验原理}
    \subsection{声速公式}
    在理想空气中,声波的传播速度满足:
    $$
        v=\sqrt{\dfrac{\gamma R T}{M}}
    $$

    忽略空气中水蒸气的影响和其他夹杂物的影响,在开式温度$T$下,声速满足:
    $$
        v = v_0 \sqrt{T / 273.15}
    $$

    其中,$v_0$为$0^\circ C$理想空气中的声速,计算得到的值为$331.45 m/s$

    \subsection{测量方法}

    根据波动理论:
    $$
        v = \lambda \cdot f
    $$

    其中$v$为波速,$\lambda$为波长,$f$为频率。声波的频率由声源的电激励信号给出,波长由共振干涉法(驻波假设)和相位比较法(行波假设)来测量。

    共振干涉法:在接受表面,入射波和反射波的振动方向与频率相同而发生相干叠加;声波在发射表面和接受表面之间往返,形成驻波。当接受表面和发射表面之间的距离为半波长$\lambda /2$的整数倍时,出现稳定的驻波共振现象,声压最大。

    相干比较法,发射端和接收端之间存在相位差,可以通过李萨如图形来观察。当$S1$和$S2$之间距离变化半个波长,图形变化$\Delta \varphi = \pi$

    \section{实验方法}

    组装实验仪器。

    首先测量环境温度$T$、和简谐频率$f$:在接收平面和发射平面之间保证一定距离的前提下,调整正弦信号频率,当在某一点电压幅度最大时,记录该频率为匹配点频率$f$。

    共振干涉法(驻波法)测量空气中的波长和声速:保持接收平面和发射平面距离$>5cm$,通过移动鼓轮调整接收平面位置,观察示波器,当示波器上出现振幅最大信号时,记录接收平面的位置$L$,按照一定方向改变接收器的位置,可以发现波形发生周期性的变化,共记录至少$12$个为支点,并用最小二乘法处理数据,并计算不确定度;与由温度计算得到的理论值相比。

    相位干涉法(行波法)测量水中的波长和声速:调整接线方式,并将示波器置于$X-Y$挡位上,保持接收平面和发射平面距离$>5cm$,通过移动鼓轮调整接收平面位置,观察示波器,当示波器上出现斜率为正(或负)的直线时,记录接收平面的位置$L$,用同样的方法共记录至少$8$个为支点,并用最小二乘法处理数据,并计算不确定度;与由温度计算得到的理论值相比。

    测量固体中的声速的大小(时差法):分别将黄铜棒、有机玻璃棒连接在实验仪器上,调整接线方式,测出时间差$t$;用游标卡尺测量并记录长度$L$

    \section{实验数据}

    共振干涉法(驻波法)数据$L$:($f = 37121 Hz,T= 23.0^{\circ}C$)
    $$
        \begin{aligned}
             & 29.395,28.934,28.416,28.010, 27.518,27.092, \\&26.635,26.083,25.572,25.625,23.786,22.869,\\&22.377,21.837,21.385\\
        \end{aligned}
    $$

    \bigskip
    相位干涉法(行波法)数据$L$:($f = 34787 Hz$)
    $$
        \begin{aligned}
             & 8.454,11.156,13.238,15.142, \\&17.404,19.626,21.838,21.130\\
        \end{aligned}
    $$

    \bigskip
    黄铜棒:$t=80\upmu s$,$L=25.766 cm$;有机玻璃棒:$t=154\upmu s$,$L=26.776m$。

    \section{数据分析}
    \subsection{共振干涉法(驻波法)}
    注:在分析数据时注意到,数据点$10,11,12$中间存在明显偏差,应该是漏掉了三个波峰,通过调整最小二乘的$n$可以修正这些错误。

    对数据分别编号$n=1,2,3\ldots$,利用最小二乘得到回归方程:

    \smallskip
    $L = -0.47034581\times n + 29.86871233$

    \smallskip

    通过逐差的方法,对$L$进行不确定度分析:
    $$
        U_{AL} = \sqrt{\frac{1}{n-1} \sum_{i=1}^{n-1} (\Delta L_{i} -\bar{\Delta L})^2} = 0.09269 cm
    $$

    仪器的误差:$U_{BL} =\sqrt 3 \times \Delta = 0.001732cm$

    \smallskip
    因此$L$的展伸不确定度:$U_{L} = \sqrt{U_{AL}^2 + U_{BL}^2} = 0.09271 cm$

    \smallskip
    $\Delta f = 1Hz$,不确定度:$U_{f} =1Hz$

    \smallskip
    因此$U_{v} / v=\sqrt{(\frac {U_{L}} {L})^{2} + (\frac {U_{f}} {f})^{2}} = 0.003616$

    \smallskip
    得到半波长$\lambda / 2 = 0.470 cm$,计算得到声速$v = 349.1\pm 1.3 m/s,P=0.68$;

    \smallskip
    对比理论值$v_0 = 345.12 m/s$,相对误差$1.17\%$

    \subsection{相干干涉法(行波法)}
    用与上文类似的处理方法可以得到:

    $$
        \begin{aligned}
             & f = 34787 Hz \pm 1 Hz           \\
             & \lambda = 4.394 cm \pm 0.001 cm \\
        \end{aligned}
    $$

    所以测得水中的声速:$v = 1528.7 m/s \pm 1\times 10^{-5} m/s$

    \subsection{测量固体中的声速的大小(时差法)}

    限于时间原因,并未测量多组数据。

    \smallskip
    黄铜棒:
    $$
        \begin{aligned}
             & t = 80\ \upmu s \pm 1\ \upmu s \\
             & L = 25.776 \pm 0.001 cm        \\
        \end{aligned}
    $$

    得到黄铜棒中声速为:$v= 3222 m/s \pm 1 m/s$

    \smallskip
    有机玻璃棒:
    $$
        \begin{aligned}
             & t = 154\ \upmu s \pm 1\ \upmu s \\
             & L = 26.776 \pm 0.001 cm         \\
        \end{aligned}
    $$

    得到黄铜棒中声速为:$v= 1739 m/s \pm 1 m/s$

    \section{思考题}
    \begin{itemize}
        \item 定性分析共振法测量时,声压振幅极大值随着距离变长而减小的原因。

              声波的传递有损,振动幅度受能量减小的影响而减小
        \item 声速测量中驻波法、相位法、时差法有何异同?

              驻波法测量精度更高,耗时也更长,适合精密测量。但缺点在于不直观、不容易判断最大振幅的位置;相位法在示波器上更加直观,更容易辨别最大点,但行波假设并不适用于大多数条件;时差法更加迅速,但精度不高。
        \item 各种气体中的声速是否相同,为什么?

              呃,不相同,不同种气体之间定压比热容和定容比热容不相同……它们之间的比值也不相同;实际测量时固体中的声速不相同也可以从侧面反映出:声音的传播不仅受介质形态的影响,也受介质本身性质的影响。
    \end{itemize}
\end{multicols}
\end{document}