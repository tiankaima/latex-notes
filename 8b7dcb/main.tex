% !TeX encoding = UTF-8
% !TeX program = xelatex
% !TeX spellcheck = en_US

\documentclass[a4paper]{ltxdoc}
\usepackage{amsmath}
\usepackage[UTF8]{ctex}
\usepackage{unicode-math}
\usepackage{caption}
\usepackage{booktabs}
\usepackage{xcolor}
\usepackage{array}
\usepackage{listings}
\usepackage[perpage]{footmisc}
\usepackage{hypdoc}
\usepackage{geometry}
\usepackage{endnotes}
\usepackage{graphicx}
% \usepackage[multiple]{endnotes}
\usepackage{multicol}
\usepackage{blindtext}
\geometry{a4paper, scale=0.85}
\newenvironment{Figure}
{\par\medskip\noindent\minipage{\linewidth}}
{\endminipage\par\medskip}

\title{实验报告\\衍射实验}
\author{少年班学院\\马天开 PB21000030(9号)}
\date{\today}
% TODO: 稍微调整文字内容。

\begin{document}
\begin{multicols}{2}
    \maketitle
    \section{实验目的}
    对光学实验形成初步认识,熟悉基本概念,掌握组装、调整实验仪器的方法。

    观察夫琅禾费衍射现象,研究不同结构衍射屏的光强分布特征,对夫琅禾费衍射图样的条纹进行测量。

    结合理论计算衍射屏的结构参数(单缝的宽度,双缝中心间距,小孔直径)
    \section{实验器材}
    光学导轨、电源、\textit{He-Ne}激光器($\lambda = 632.8 nm$)、衍射元件、\textit{CCD}、显示器。
    \section{实验原理}
    当光源和接收屏都离衍射屏足够远的时候,即菲涅耳数$F\triangleq \dfrac{a^2}{L\lambda} \ll 1$时,在接收屏产生的衍射即为夫琅禾费衍射。

    其中$a$是孔径的尺寸,$L$是孔径与观察屏之间的距离,$\lambda$是入射波的波长。

    由标量衍射理论,有基尔霍夫衍射积分公式:

    $$
        \widetilde{U}(P) = -\dfrac{1}{2} \oiint _{\Sigma} \dfrac{1}{2}(\cos \theta_0 + \cos \theta)\widetilde{U_0}(Q)\dfrac{e^{ikr}}{r} dS
    $$

    利用上述公式,适当近似之后,可以得到特殊种类的衍射光强分布。

    \subsection{单缝夫琅禾费衍射}

    单缝夫琅禾费衍射的光强分布为:

    $$
        \left\{
        \begin{aligned}
            I(\phi) & = I_0 (\dfrac{\sin u}{u})^2                  \\
            u       & \triangleq \dfrac {\pi a \sin\phi} {\lambda} \\
        \end{aligned}
        \right.
    $$

    其中 $a$为单缝的宽度,$I_0$为入射光光强,$\phi$为衍射光与光轴的夹角(衍射角)。

    当$\phi = 0$时,光强取最大值$I_0$,对应的衍射条纹为中央大条纹;当$a\sin\phi=k\lambda,k=\pm 1,\pm 2,\pm 3,\ldots$时,$I(\phi)=0$,此时为暗条纹。

    做小角度近似$\sin\phi\approx\phi\approx\tan\phi$,因此有:

    $$
        \dfrac{k\lambda}{a} = \dfrac{x_k}{L}
    $$

    其中$x_k$为第$k$条暗条纹与中央亮条纹的距离。

    \subsection{双缝夫琅禾费衍射}

    双缝夫琅禾费衍射的光强分布为:

    $$
        \left\{
        \begin{aligned}
            I(\phi) & = 4I_0 (\dfrac{\sin u}{u})^2 \cos^2 v        \\
            u       & \triangleq \dfrac {\pi a \sin\phi} {\lambda} \\
            v       & \triangleq \dfrac {\pi d \sin\phi} {\lambda} \\
        \end{aligned}
        \right.
    $$

    其中$a$为单缝的宽度,$d$为双缝间距,$I_0$为入射光光强,$\phi$为衍射光与光轴的夹角(衍射角)。

    当$\phi = 0$时,光强取最大值$I_0$,对应的衍射条纹为中央大条纹;当$v=n\pi$时,即$d\sin\phi=n\lambda,n=\pm 1,\pm 2,\pm 3\ldots$,此时为双缝干涉的极大值。

    当$\dfrac{n\lambda}{d} = \dfrac{x_n}{L}$,其中$x_n$为第$n$级暗条纹与中央亮条纹的距离。

    若上面两个式子($d\sin\phi =n\lambda$与$a\sin\phi =k\lambda$)位置相重合,此时第$n$级干涉极大将不会出现,称为缺级。

    \subsection{圆孔夫琅禾费衍射}

    圆孔夫琅禾费衍射的光强分布为:

    $$
        \left\{
        \begin{aligned}
            I(\theta) & = I_0 (\dfrac{2J_1(m)}{m})^2        \\
            m         & = \dfrac{2\pi R\sin\theta}{\lambda} \\
        \end{aligned}
        \right.
    $$

    其中$R$为圆孔,$J_1(m)$为一阶\textit{Bezier}函数。

    当$m_1=3.83$处,对应角半径为$\theta_1\approx\sin\theta_1=\dfrac{3.83\lambda}{2\pi R}=\dfrac{0.61\lambda}{R}$,即为零级亮斑。

    % TODO: Relative position require noticing.
    \bigskip

    因此,艾里斑的半角宽度为:

    $$
        \theta_0 = 1.22 \dfrac{\lambda}{D}
    $$

    \section{实验方法}

    \begin{itemize}
        \item 调节导轨及\textit{CCD}:

              依次装好激光器、衰减片、衍射原件支架和\textit{CCD}镜头,并保持它们的中心位置在一条与导轨平行的直线上。

        \item 将衍射元件置于支架上,接通装置中所有电源,调整衍射元件的位置,使光路通过元件待观察部分。调整衰减片相对位置,使得照在\textit{CCD}的光强处在合适水平,去掉\textit{CCD}上的保护盖,待显示屏稳定之后观察现象。

        \item 观察单缝、双缝、小孔衍射光强分布,总结各元件衍射图样的特点
        \item 观察并总结单缝、双缝、小孔缝宽(或直径)与衍射图样变化之间的关系
        \item 记录单缝衍射各级暗条纹及中央主极大位置,计算单缝宽$a$,求相对误差
              % \item 记录双缝衍射各级亮条纹(或暗条纹)位置,计算双缝中心间距$d$($d=a+b$,为光栅常数),求相对误差
    \end{itemize}
\end{multicols}
\section{实验数据}
\begin{tabular}{|c|c|c|c|c|c|c|c|}
    \hline 单缝宽度 & 衍射元件位置读数 & \textit{CCD}位置读数 & \textit{CCD}镜头凸出 & 中心位置 & 左1    & 左2    & 左3    \\
    \hline 0.2      & 59.00            & 89.21                & 2.00                 & 14.186   & 13.268 & 12.432 & 11.443 \\
    \hline 0.2      & 59.00            & 89.21                & 2.00                 & 14.165   & 13.278 & 12.437 & 11.463 \\
    \hline 0.2      & 59.00            & 89.21                & 2.00                 & 14.146   & 13.268 & 12.396 & 11.462 \\
    \hline 0.1      & 59.00            & 89.21                & 2.00                 & 14.416   & 12.487 & 10.705 & 8.872  \\
    \hline 0.1      & 59.00            & 89.21                & 2.00                 & 14.350   & 12.498 & 10.709 & 8.878  \\
    \hline 0.1      & 59.00            & 89.21                & 2.00                 & 14.345   & 12.475 & 10.676 & 8.793  \\
    \hline 0.05     & 59.00            & 89.21                & 2.00                 & 14.352   & 11.055 & -      & -      \\
    \hline 0.05     & 59.00            & 89.21                & 2.00                 & 14.269   & 11.080 & -      & -      \\
    \hline 0.05     & 59.00            & 89.21                & 2.00                 & 14.215   & 11.110 & -      & -      \\\hline
    \hline 0.2      & 75.61            & 89.21                & 2.00                 & 14.328   & 14.000 & -      & -      \\
    \hline 0.2      & 75.61            & 89.21                & 2.00                 & 14.365   & 13.395 & -      & -      \\
    \hline 0.2      & 75.61            & 89.21                & 2.00                 & 14.351   & 13.975 & -      & -      \\
    \hline 0.1      & 75.61            & 89.21                & 2.00                 & 14.537   & 13.725 & -      & -      \\
    \hline 0.1      & 75.61            & 89.21                & 2.00                 & 14.523   & 13.520 & -      & -      \\
    \hline 0.1      & 75.61            & 89.21                & 2.00                 & 14.538   & 13.719 & -      & -      \\
    \hline 0.05     & 75.61            & 89.21                & 2.00                 & 14.453   & 13.136 & -      & -      \\
    \hline 0.05     & 75.61            & 89.21                & 2.00                 & 14.475   & 13.125 & -      & -      \\
    \hline 0.05     & 75.61            & 89.21                & 2.00                 & 14.458   & 13.132 & -      & -      \\\hline
    \hline 0.2      & 54.82            & 89.21                & 2.00                 & 13.761   & 12.680 & -      & -      \\
    \hline 0.2      & 54.82            & 89.21                & 2.00                 & 13.750   & 12.897 & -      & -      \\
    \hline 0.2      & 54.82            & 89.21                & 2.00                 & 13.703   & 12.681 & -      & -      \\
    \hline 0.1      & 54.82            & 89.21                & 2.00                 & 14.339   & 12.198 & -      & -      \\
    \hline 0.1      & 54.82            & 89.21                & 2.00                 & 14.338   & 12.212 & -      & -      \\
    \hline 0.1      & 54.82            & 89.21                & 2.00                 & 14.351   & 12.228 & -      & -      \\
    \hline 0.05     & 54.82            & 89.21                & 2.00                 & 14.316   & 10.778 & -      & -      \\
    \hline 0.05     & 54.82            & 89.21                & 2.00                 & 14.461   & 10.770 & -      & -      \\
    \hline 0.05     & 54.82            & 89.21                & 2.00                 & 14.382   & 10.807 & -      & -      \\\hline
\end{tabular}
% \begin{multicols}{2}
\section{实验结果}
利用$\dfrac{k\lambda}{a} = \dfrac{x_k -x_0}{L}$,可以计算出(其中大部分数据无法线性拟合):
\bigskip

\begin{tabular}{|c|c|c|c|c|}
    \hline 单缝宽度 & 衍射元件位置读数 & \textit{CCD}位置读数 & $\bar{a} 计算值$ & 相对误差 \\
    \hline 0.2      & 59.00            & 89.21                & 0.1932           & 3.4\%    \\
    \hline 0.1      & 59.00            & 89.21                & 0.1023           & 2.3\%    \\
    \hline 0.05     & 59.00            & 89.21                & 0.04872          & 2.6\%    \\\hline

    \hline 0.2      & 75.61            & 89.21                & 0.1956           & 1.2\%    \\
    \hline 0.1      & 75.61            & 89.21                & 0.1114           & 1.1\%    \\
    \hline 0.05     & 75.61            & 89.21                & 0.05061          & 1.2\%    \\\hline

    \hline 0.2      & 54.82            & 89.21                & 0.2103           & 5.2\%    \\
    \hline 0.1      & 54.82            & 89.21                & 0.0923           & 7.7\%    \\
    \hline 0.05     & 54.82            & 89.21                & 0.04974          & 0.5\%    \\\hline
\end{tabular}
\section{实验分析}
\subsection{总结}
本次实验观察了各种元件夫琅禾费衍射图像的观察,以及单缝、双缝的测量,误差主要来源于条纹中心的不确定性,产生了较大偏差。
\subsection{思考题}
\begin{itemize}
    \item 当光通过一个小孔时,在后面的光屏会得到什么样的图案?
          当小孔孔径较小时,与圆孔夫琅禾费衍射图案类似。当孔径稍微更大时,在边缘处会产生衍射现象。
    \item 白光照射到狭缝上,衍射条纹有什么特点?
          为多种波长衍射条纹的叠加图案,除中央大条纹,其他不存在全波长的极大值。
    \item \textit{LED}射灯照到手机屏幕可观察到特殊的图案,解释其原因。
          \textit{LED}照射到手机屏幕时,会分别在上下表面分别折射;特别的,在像素点上照射会产生类似圆孔夫琅禾费衍射的图案,多种波长衍射图案叠加在一起,即为图中所示图案。
\end{itemize}
% \end{multicols}
\end{document}