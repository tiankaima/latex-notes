% !TeX encoding = UTF-8
% !TeX program = xelatex
% !TeX spellcheck = en_US

\documentclass[a4paper]{ltxdoc}
\usepackage{amsmath}
\usepackage[UTF8]{ctex}
\usepackage{unicode-math}
\usepackage{caption}
\usepackage{booktabs}
\usepackage{xcolor}
\usepackage{array}
\usepackage{listings}
\usepackage[perpage]{footmisc}
\usepackage{hypdoc}
\usepackage{geometry}
\usepackage{endnotes}
% \usepackage[multiple]{endnotes}
\usepackage{multicol}
\usepackage{blindtext}
\geometry{a4paper, scale=0.85}


\title{实验报告\\单摆法测量重力加速度}
\author{少年班学院\\马天开 PB21000030(第三组)}
\date{\today}

\begin{document}
\begin{multicols}{2}
  \maketitle
  \section{实验背景}
  重力加速度,也称自由落体加速度,表示物体受重力影响产生的加速度。20世纪70年代初,国际上建立了国际标准重力网(IGSN)
  \endnote{abbr. for International Gravity Standardization Network}。
  IGSN的平均精度优于$10^{-7}m/s^2$;近年来,也出现了一些更精确测量的办法
  \endnote{Zong, B.C., Nie, C.H., 2013. Measuring the Gravitational Acceleration by Hydrostatics. Applied Mechanics and Materials 307, 271–274.. doi:10.4028/www.scientific.net/amm.307.271}。
  精确地测量$g$的值,对军工、卫星等有着重要的意义。
  \section[实验目的]{实验目的\endnote{摘自实验要求:重力加速度的测量.pdf}}
  在给定器材和对重力加速度$g$的测量精度要求($< 10^{-2}$)下,根据单摆公式,进行简单的实验设计和实验基本方法的训练,学会实验器材的使用、测量方法和应用误差均分原理;分析误差的来源,提出修正和估算的方法。
  \section{实验器材}
  游标卡尺、米尺、电子秒表、支架、细线、小刚球、摆幅标尺

  游标卡尺精度:$\Delta\approx 10^{-3}cm$,米尺精度:$\Delta\approx 0.05cm$,秒表精度:$\Delta\approx 10^{-2}s$,人的反应时间:$\Delta\approx 0.1s$
  \section{实验原理}
  根据牛顿第二定律:
  \begin{equation}
    m\cdot l\cdot \frac {d^2 \theta}{dt^2} = -mg\sin\theta
  \end{equation}

  简单整理之后,方程变为一个二阶微分方程:
  \begin{equation}
    \frac {d^2 \theta}{dt^2} = -\frac g l \sin\theta \label{XX}
  \end{equation}

  将右侧$\sin\theta$在$\theta = 0$按Maclaurin展开:
  \begin{equation}
    \frac {d^2 \theta}{dt^2} = -\frac g l (\theta - \frac {\theta^3} 6 +\cdots)
  \end{equation}

  考虑到$\theta - \sin \theta<\frac {\theta^3} 6$,在$\theta < 10 ^{\circ}$时,$\Delta = (\theta - \sin\theta)/\theta < 0.005$,符合精度要求。
  因此将\eqref{XX}中的$\sin\theta$替换为$\theta$,近似为二阶线性微分方程,代入初值条件:
  $$\left\{
    \begin{aligned}
      \frac {d\theta} {dt} \mid _{t = 0} & = 0        \\
      \theta (0)                         & = \theta_0
    \end{aligned}
    \right.$$
  解出:
  \begin{equation}
    \begin{aligned}
      \theta(t) & =\theta_0\cdot\cos (\omega t+\phi)                  \\
      T         & = \frac {2\pi}{\omega} = 2\pi \cdot\sqrt{\frac l g}
    \end{aligned}
  \end{equation}
  这给出了一个计算$g$的方法
  \endnote{以上计算与实际情况的误差还会存在于:摆线可能的长度变化、摆线的质量、受空气的影响。以上修正项在精度$10^{-3}$以内可以忽略。}:
  \begin{equation}
    g = l \cdot (\frac {2\pi} T)^2 \label{XY}
  \end{equation}
  \section{实验方法}
  首先用游标卡尺测量小球的直径$d$,尺寸约为$2cm$,估算误差$\Delta d /d\approx 0.5 \times 10^{-3}$在允许范围内。

  接着测量摆线长度:用米尺测量从单摆的顶端开始,到小球的顶端的距离,尺寸约在$60cm$,估算误差$\Delta l^{\prime} /l^{\prime} \approx 1.2 \times 10^{-4}$在允许范围内。

  注意这里测量的摆线长度并不等同于摆长,后续计算的摆长$l=l^{\prime} + d/2$

  最后是测量摆动周期$T$。这里值得注意的是,如果只测量一个周期的时间(估计为$1s$),误差会达到$\Delta t/t\approx 0.2$,远远超过实验允许的误差$10^{-3}$。因此,选择测量100个周期所需时间,误差约为$\Delta t/t\approx 2\times 10^{-3}$,在允许范围内。与之相比,秒表测量时的误差可以忽略不记\endnote{约为 $10^{-6}$量级}。

  总体的系统误差估计:
  $$\lvert\Delta g/g\rvert \leq \lvert\Delta l/l\rvert + \lvert2\times \Delta t/t\rvert\\
    \approx 2\times 10^{-3}$$

  \section[实验数据]{实验数据\endnote{实际测量时,出于时间上的考虑,前两次实验时并未测量100个周期,而是选用了55个周期,在保证精度的前提下完成了实验。}}
  \begin{tabular}{|c|c|c|c|}
    % 摆长                & 角度              & 测量次数     & 测量结果      \\
    \hline\textbf{$l$} & \textbf{$\theta$} & \textbf{$n$} & \textbf{$nT$} \\
    \hline $60.273cm$  & $10^{\circ}$      & $55$         & $85.85$       \\
    \hline $60.172cm$  & $5^{\circ}$       & $55$         & $85.81$       \\
    \hline $60.133cm$  & $15^{\circ}$      & $100$        & $154.94$      \\
    \hline $49.878cm$  & $10^{\circ}$      & $100$        & $141.93$      \\
    \hline
  \end{tabular}

  \bigskip
  签字的实验数据(复印件)在附件中。
  \section{数据处理}
  \subsection{不确定度分析}

  实验中的系统误差主要来源于\endnote{按照\eqref{XY} $T$应该与$l$相关,因此只计算前三次的数据。}:
  \begin{itemize}
    \item 测量$l$产生的误差:

          $l$的标准差:
          $$
            \sigma_l = \sqrt{\frac {\sum _{i=1}^{n} (l_i - \bar l)^2} {n-1} }= 0.072 cm
          $$

          展伸不确定度:($\Delta_b$ 是米尺的精度)
          $$
            \begin{aligned}
              U_l & =\sqrt{(t_{0.68} \frac {\sigma _l} {\sqrt{n}})^2 + (k_b \frac {\Delta_b} {C})^2} = 0.05 cm, \\
              P   & =0.68
            \end{aligned}
          $$
    \item 测量$T$产生的误差

          $T$的标准差:
          $$
            \sigma_T = \sqrt{\frac {\sum _{i=1}^{n} (l_i- \bar l)^2} {n-1}} = 0.006 s
          $$

          展伸不确定度:($\Delta_T$ 是秒表的精度)
          $$
            \begin{aligned}
              U_T & = \sqrt{(t_{0.68} \frac {\sigma _T} {\sqrt{n}})^2 + (k_b \frac {\Delta_T} {C})^2} = 0.013 s, \\
              P   & =0.68
            \end{aligned}
          $$

          由以上内容可以得到$g$的展伸不确定度:

          $$
            \begin{aligned}
              \frac {U_g}{\bar g} & = \sqrt {(\frac {U_l} {\bar L}) + 2(\frac{U_T}{\bar T})^2} = 0.0094, \\
              P                   & =0.68
            \end{aligned}
          $$

          其中$U_g/\bar g=0.0094 < 0.01$ 满足实验预期。
  \end{itemize}
  \subsection{数值计算}
  按照\eqref{XY}中给出的公式,计算$g$的值分别为:
  $$
    \left\{
    \begin{aligned}
      g_1 & =9.766 m/s^2 \\
      g_2 & =9.759 m/s^2 \\
      g_3 & =9.889 m/s^2 \\
      g_4 & =9.775 m/s^2 \\
    \end{aligned}
    \right.
  $$

  前三次测量结果取平均的结果:
  $$
    \left\{
    \begin{aligned}
      \bar T   & = 1.557 s \\
      \ \bar l & = 0.602 m
    \end{aligned}
    \right.
  $$

  按这个结果计算得到:
  $$\bar g=\bar l\cdot (\frac {2\pi} {\bar T})^2=9.803m/s^2$$

  但这个办法并没有利用第四组数据,可以考虑下面的办法:

  利用\eqref{XY}进行线性回归分析,对\eqref{XY}两侧取对数:
  \begin{equation}
    \ln l - 2\ln T = \ln g - 2\ln(2\pi)
  \end{equation}

  以$\ln l\ -\ \ln T$进行拟合,得到:
  $$
    \ln l= 2.022\cdot \ln T -1.403
  $$

  利用$b=\ln g -2\cdot \ln(2\pi)$推算$\widetilde{g}$的值:
  $$
    \widetilde{g} = 9.71m/s^2
  $$

  由于利用了四个测量值,采纳这个结果作为测量结果。

  \subsection{结论}
  综合上面的讨论,最终测量的结果为:
  $$
    g=\widetilde{g} \pm \bar U_g =9.71 \pm 0.09 m/s^2, P=0.68
  $$
  \theendnotes

  % \section{改进方案}

  % 单摆法测量重力加速度,受限于公式的近似,测量结果精度一般小于$10^{-4}$,改进空间不大。

  % 可以考虑将测量的周期数加长至$200$次,可以将精度提高到$10^{-2} m /s^2$
\end{multicols}
\end{document}