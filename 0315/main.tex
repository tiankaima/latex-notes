% !TeX encoding = UTF-8
% !TeX program = xelatex
% !TeX spellcheck = en_US

\documentclass[a4paper]{ltxdoc}
\usepackage{amsmath}
\usepackage[UTF8]{ctex}
\usepackage{unicode-math}
\usepackage{caption}
\usepackage{booktabs}
\usepackage{xcolor}
\usepackage{array}
\usepackage{listings}
\usepackage[perpage]{footmisc}
\usepackage{hypdoc}
\usepackage{geometry}
\usepackage{endnotes}
% \usepackage[multiple]{endnotes}
\usepackage{multicol}
\usepackage{blindtext}
\geometry{a4paper, scale=0.85}


\title{实验报告\\单摆法测量重力加速度}
\author{少年班学院\\马天开 PB21000030(第三组)}
\date{\today}

\begin{document}
\maketitle
\begin{multicols}{2}
  \section{实验背景}
  重力加速度,也称自由落体加速度,表示物体受重力影响产生的加速度。20世纪70年代初,国际上建立了国际标准重力网(IGSN)
  \endnote{abbr. for International Gravity Standardization Network}。
  IGSN的平均精度优于$10^{-7}m/s^2$;近年来,也出现了一些更精确测量的办法
  \endnote{Zong, B.C., Nie, C.H., 2013. Measuring the Gravitational Acceleration by Hydrostatics. Applied Mechanics and Materials 307, 271–274.. doi:10.4028/www.scientific.net/amm.307.271}。
  精确地测量$g$的值,对军工、卫星等有着重要的意义。
  \section{实验目的}
  在给定器材和对重力加速度$g$的测量精度要求($< 10^{-2}$)下,根据单摆公式,进行简单的实验设计和实验基本方法的训练,学会实验器材的使用、测量方法和应用误差均分原理;分析误差的来源,提出修正和估算的方法。
  \section{实验器材}
  游标卡尺、米尺、电子秒表、支架、细线、小刚球、摆幅标尺

  游标卡尺精度:$\Delta\approx 10^{-3}cm$,米尺精度:$\Delta\approx 0.05cm$,秒表精度:$\Delta\approx 10^{-2}s$,人的反应时间:$\Delta\approx 0.1s$
  \section{实验原理}
  根据牛顿第二定律:
  \begin{equation}
    m\cdot l\cdot \frac {d^2 \theta}{dt^2} = -mg\sin\theta
  \end{equation}

  简单整理之后,方程变为一个二阶微分方程:
  \begin{equation}
    \frac {d^2 \theta}{dt^2} = -\frac g l \sin\theta \label{XX}
  \end{equation}

  将右侧$\sin\theta$在$\theta = 0$按麦克劳林展开:
  \begin{equation}
    \frac {d^2 \theta}{dt^2} = -\frac g l (\theta - \frac {\theta^3} 6 +\cdots)
  \end{equation}

  考虑到$\theta - \sin \theta<\frac {\theta^3} 6$,在$\theta < 10 ^{\circ}$时,$\Delta = (\theta - \sin\theta)/\theta < 0.005$,符合精度要求。
  因此将\eqref{XX}中的$\sin\theta$替换为$\theta$,近似为二阶偏微分方程,代入初值条件:
  $$\left\{
    \begin{aligned}
      \frac {d\theta} {dt} \mid _{t = 0} & = 0        \\
      \theta (0)                         & = \theta_0
    \end{aligned}
    \right.$$
  解出:
  \begin{equation}
    \begin{aligned}
      \theta(t) & =\theta_0\cdot\cos (\omega t+\phi)                  \\
      T         & = \frac {2\pi}{\omega} = 2\pi \cdot\sqrt{\frac l g}
    \end{aligned}
  \end{equation}
  这给出了一个计算$g$的方法
  \endnote{以上计算与实际情况的误差还会存在于:摆线可能的长度变化、摆线的质量、受空气的影响。以上修正项在精度$10^{-3}$以内可以忽略。}:
  \begin{equation}
    g = l \cdot (\frac {2\pi} T)^2 \label{XY}
  \end{equation}
  \section{实验方法}
  首先用游标卡尺测量小球的直径,尺寸约为$2cm$,误差$\Delta r /r\approx 0.5 \times 10^{-3}$在允许范围内。

  接着用米尺测量从单摆的顶端开始,到小球的顶端的距离
  \endnote{为准确获取模型中的摆长},尺寸约在$60cm$,误差$\Delta l /l \approx 1.2 \times 10^{-4}$在允许范围内。

  这里值得注意的是,如果只测量一个周期的时间(估计为$1s$),误差会达到$\Delta t/t\approx 0.2$,远远超过实验允许的误差$10^{-3}$。因此,选择测量100个周期所需时间,误差约为$\Delta t/t\approx 2\times 10^{-3}$,在允许范围内。与之相比,秒表自身的误差可以忽略不记\endnote{约为 $10^{-6}$量级}。

  总体的系统误差估计:
  $$\lvert\Delta g/g\rvert \leq \lvert\Delta l/l\rvert + \lvert2\times \Delta t/t\rvert\\
    \approx 2\times 10^{-3}$$
  \section[实验数据]{实验数据\endnote{实际测量时,出于时间上的考虑,前两次实验时并未测量100个周期,而是选用了55个周期,在保证精度的前提下完成了实验。}$^{,}$\endnote{实验过程中,我们认为:在未调节摆长的前提下,多次测量摆长的操作是无意义的;甚至会降低实验结果的准确性。因为这并不能提供令人满意的误差分布。}}
  \begin{tabular}{|c|c|c|c|}
    \hline \textbf{摆长 $l$} & \textbf{角度 $\theta$} & \textbf{测量次数} & \textbf{测量结果$t$} \\
    \hline $60.000cm$        & $10^{\circ}$           & $55$              & $85.85$              \\
    \hline $60.000cm$        & $5^{\circ}$            & $55$              & $85.81$              \\
    \hline $60.000cm$        & $15^{\circ}$           & $100$             & $154.94$             \\
    \hline $49.878cm$        & $10^{\circ}$           & $100$             & $141.93$             \\
    \hline
  \end{tabular}
  \section{数据处理}
  \subsection{不确定度分析}
  我们认为在本实验中多次测量同一变量的样本偏小,因而偏差更大。因而并没有进行重复实验。

  实验中的系统误差主要来源于:
  \begin{itemize}
    \item 测量$l$产生的误差
    \item 测量$t$产生的误差,一篇论文详细地测定了实验者接受到视觉反馈到秒表停止计时的误差分布\endnote{Faux, D.A., Godolphin, J., 2019. Manual timing in physics experiments: Error and uncertainty. American Journal of Physics 87, 110–115.. doi:10.1119/1.5085437};我们采纳其中的分布$N(0.11,0.072)$,并考虑系统误差
  \end{itemize}
  \subsection{数值计算}
  按照\eqref{XY}中给出的公式,计算$g$的值分别为:
  \begin{equation}
    \left\{
    \begin{aligned}
      g_1 & =9.722\pm 0.03\ m/s^2 \\
      g_2 & =9.731\pm 0.03\ m/s^2 \\
      g_3 & =9.867\pm 0.03\ m/s^2 \\
      g_4 & =9.775\pm 0.03\ m/s^2 \\
    \end{aligned}
    \notag{}
    \right.
  \end{equation}

  前三次测量结果取平均的结果:
  \begin{equation}
    \left\{
    \begin{aligned}
      \bar T & = \frac 1 3(\frac{85.85}{55}+\frac{85.81}{55}+\frac{154.94}{100})=1.557 \pm 0.001\ s \\
      \bar l & = 0.600 m \pm  0.001\ m                                                              \\
    \end{aligned}
    \notag
    \right.
  \end{equation}

  按这个结果计算得到\endnote{只计算了前三项的数据}:
  $$\bar g=\bar l\cdot (\frac {2\pi} {\bar T})^2=9.77\pm 0.03\ m/s^2$$

  或者,可以利用\eqref{XY}进行线性回归分析,对\eqref{XY}两侧取对数:
  \begin{equation}
    \ln g = \ln l + 2\cdot(\ln(2\pi) -\ln(T))
  \end{equation}

  对$\ln l\ -\ \ln T$,拟合得到:
  \begin{equation}
    \begin{aligned}
      ln l= & 1.9871912225569373\cdot \ln T \\
            & -1.3906943077151763           \\
    \end{aligned}
    \notag{}
  \end{equation}

  回归系数$a=1.98\ldots$与公式推导的$2$之间误差符合预期,通过$b=\ln g -2\cdot \ln(2\pi)$推出$g$的值:
  \begin{equation}
    g = 9.83\pm 0.03\ m/s^2
  \end{equation}
  
  \theendnotes
\end{multicols}
\end{document}