% !TeX encoding = UTF-8
% !TeX program = xelatex
% !TeX spellcheck = en_US

\documentclass[a4paper]{ltxdoc}
\usepackage{amsmath}
\usepackage[UTF8]{ctex}
\usepackage{unicode-math}
\usepackage{caption}
\usepackage{booktabs}
\usepackage{xcolor}
\usepackage{array}
\usepackage{listings}
\usepackage[perpage]{footmisc}
\usepackage{hypdoc}
\usepackage{geometry}
\usepackage{endnotes}
% \usepackage[multiple]{endnotes}
\usepackage{multicol}
\usepackage{blindtext}
\geometry{a4paper, scale=0.85}
\newenvironment{Figure}
  {\par\medskip\noindent\minipage{\linewidth}}
  {\endminipage\par\medskip}

\title{实验报告\\切变模量的测量}
\author{少年班学院\\马天开 PB21000030(1号)}
\date{\today}


\begin{document}
\begin{multicols}{2}
  \maketitle
  \section{实验目的}
  利用扭摆测量金属丝的切变模量,根据公式进行简单的实验设计和实验基本方法训练,学会实验仪器的使用、测量方法和应用误差均分原理;分析误差的来源,提出修正和估算的方法。
  \section{实验器材}
  扭摆、金属悬盘、金属环、待测金属丝、螺旋测微仪、游标卡尺、米尺、秒表

  螺旋测微仪精度$\Delta = 0.01 mm$,游标卡尺精度$\Delta = 0.02mm$,米尺精度$0.1cm$,秒表精度$0.01s$(人的反应时间:$0.2s$)
  \section{实验原理}
  待测金属丝是一根均匀细长的钢丝,近似为一个半径为$R$,长度为$L$的圆柱体,按照剪切胡克定律:
  \begin{equation}
    \tau = G \gamma
  \end{equation}
  其中,$\gamma$为切应变,$\tau$为切应力,$G$为材料的切变模量。

  切应变又可以表示为:
  \begin{equation}
    \gamma = R \dfrac{d\varphi}{dl}
  \end{equation}

  在钢丝内部$r=\rho$的位置,切应变为:
  \begin{equation}
    \gamma_\rho = \rho \dfrac{d\varphi}{dl}
  \end{equation}

  在此位置会产生切应力,大小为
  \begin{equation}
    \tau_\rho = G \gamma_\rho = G\rho \dfrac{d\varphi}{dl}
  \end{equation}

  产生的回复力矩为:
  \begin{equation}
    \tau \cdot \rho \cdot 2 \pi \rho \cdot d\rho = 2\pi G \rho ^3\dfrac{d\varphi}{dl}\cdot \rho
  \end{equation}

  对$\rho$积分,得到恢复力矩:
  \begin{equation}
    \int _0 ^R 2\pi G \rho^3 d\rho \cdot \dfrac{d\varphi}{dl} = \dfrac{\pi}{2} G R ^4 \dfrac{d\varphi}{dl}
  \end{equation}

  因此,总恢复力矩($\varphi = L \dfrac{d\varphi}{L}$):
  \begin{equation}
    M = \dfrac{\pi}{2} G R ^4 \dfrac{d\varphi}{dl} = \dfrac{\pi}{2} G R ^4 \dfrac{\varphi} {L}
  \end{equation}

  \begin{equation}
    \therefore\ G = \dfrac{2ML}{\pi R^4 \varphi}
  \end{equation}

  根据转动定律:
  \begin{equation}
    M = I_0 \dfrac{d^2\varphi}{dt^2}
  \end{equation}

  又有$M = D\varphi$,因此:
  \begin{equation}
    \dfrac{d^2\varphi}{dt^2} + \dfrac{D}{I_0}\varphi = 0
  \end{equation}

  上述方程是一个简谐运动方程,周期为:
  \begin{equation}
    T_0 = 2\pi \sqrt{\dfrac{I_0}{D}}
  \end{equation}

  注意到单独计算$I_0$有困难,因此在圆盘上增加一个金属环(金属环转动惯量记为$I_1=\dfrac 1 2 m (r_1^2+r_2^2)$可以计算),测新周期为:
  \begin{equation}
    T_1 = 2\pi \sqrt{\dfrac{I_0+I_1}{D}}
  \end{equation}

  由此,可以计算:
  \begin{equation}
    G = \dfrac{4\pi Lm(r_1^2+r_2^2)}{R^4(T_1^2-T_0^2)}
  \end{equation}
  \section{实验方法}
  \begin{itemize}
    \item 调整装置,使得钢丝与圆盘相垂直,且钢丝保持悬置。
    \item 用螺旋测微器测钢丝直径(上、中、下各三组),用游标卡尺分别测量圆环的内外径(各三组)、用米尺测量钢丝的有效长度(上夹具最下端到下夹具最上端,测三组)。
    \item 考虑到人读秒时的反应时间,估算合适的测量周期数(40)。
    \item 计算切变模量$G$和扭转模量$D$,分析误差。
  \end{itemize}
  \section{实验数据}

  \smallskip
  螺旋测微器$0$示数:$-0.005mm$
  \smallskip

  \begin{tabular}{|c|c|c|c|}
    \hline \textit{$d_1$} & 0.776 & 0.778 & 0.778 \\
    \hline \textit{$d_2$} & 0.773 & 0.777 & 0.776 \\
    \hline \textit{$d_3$} & 0.781 & 0.782 & 0.784 \\\hline
  \end{tabular}

  \smallskip
  游标卡尺:
  \smallskip

  \begin{tabular}{|c|c|c|c|}
    \hline $2r_1$ & 9.988 & 9.984 & 9.988 \\
    \hline $2r_2$ & 7.948 & 7.932 & 7.958 \\\hline
  \end{tabular}

  \smallskip
  米尺:
  \smallskip

  \begin{tabular}{|c|c|c|c|}
    \hline $L$ & 44.32 & 44.38 & 44.39 \\\hline
  \end{tabular}

  \smallskip
  秒表:
  \smallskip

  \begin{tabular}{|c|c|c|c|}
    \hline $40T_0$ & 91.48  & 91.32  & 91.58  \\
    \hline $40T_1$ & 142.47 & 142.80 & 142.00 \\\hline
  \end{tabular}

  \smallskip
  天平:
  \smallskip

  $m = 512.0 g$

  \section{数据处理}

  \begin{equation}
    \left\{
    \begin{aligned}
      \bar R = \dfrac 1 {18} (0.776 + 0.778 + 0.778 + \ldots ) & = 0.394  \\
      2\bar r_1 = \dfrac{1}{3}(9.988 + 9.984 + 9.988)          & = 9.987  \\
      2\bar r_2 = \dfrac{1}{3}(7.948 + 7.932 + 7.958)          & = 7.946  \\
      \bar L = \dfrac{1}{3}(44.32 + 44.38 + 44.39)             & = 44.36  \\
      \bar T_0 = \dfrac{1}{120}(91.48 + 91.32 + 91.58)         & = 2.2865 \\
      \bar T_1 = \dfrac{1}{120}(142.47 + 142.80 + 142.00)      & = 3.561  \\
      m                                                        & = 512.0
    \end{aligned}
    \right.
  \end{equation}

  得到:$\bar G = \dfrac{4\pi Lm(r_1^2+r_2^2)}{R^4(T_1^2-T_0^2)} = 6.47 \times 10^{10}$

  % 各值的不确定度分别为:

  % \begin{equation}
  %   \left\{
  %   \begin{aligned}
  %     \sigma_a{_{R}  } & = 0.00413 mm \\
  %     \sigma_a{_{r_1}} & = 0.00117 cm \\
  %     \sigma_a{_{r_2}} & = 0.00650 cm \\
  %     \sigma_a{_{L}  } & = 0.0324 cm  \\
  %     \sigma_a{_{T_0}} & = 0.00861 s  \\
  %     \sigma_a{_{T_1}} & = 0.01007 s  \\
  %   \end{aligned}
  %   \right.
  % \end{equation}

  % 因此,$A$类不确定度分别计算为:

  各值的$A$类不确定度分别计算为:

  \begin{equation}
    \left\{
    \begin{aligned}
      u_a{_{R}  } & = 0.000947 mm \\
      u_a{_{r_1}} & = 0.000675 cm \\
      u_a{_{r_2}} & = 0.00375 cm  \\
      u_a{_{L}  } & = 0.0187 cm   \\
      u_a{_{T_0}} & = 0.00501 s   \\
      u_a{_{T_1}} & = 0.00581 s   \\
    \end{aligned}
    \right.
    \label{A}
  \end{equation}

  按照测量仪器的不同,各值的$B$类不确定度分别为:

  \begin{equation}
    \left\{
    \begin{aligned}
      u_b{_{R}  } & = 0.00289 mm \\
      u_b{_{r_1}} & = 0.00115 cm \\
      u_b{_{r_2}} & = 0.00115 cm \\
      u_b{_{L}  } & = 0.0577 cm  \\
      u_b{_{T_0}} & = 0.00577 s  \\
      u_b{_{T_1}} & = 0.00577 s  \\
      u_b{_{m}  } & = 0.0577 g   \\
    \end{aligned}
    \right.
    \label{B}
  \end{equation}

  综合~\ref{A},~\ref{B},可以计算展伸不确定度为(取$P=0.997$):

  \begin{equation}
    \left\{
    \begin{aligned}
      U_{R}   & = 0.00304 mm \\
      U_{r_1} & = 0.00133 cm \\
      U_{r_2} & = 0.00392 cm \\
      U_{L}   & = 0.0607 cm  \\
      U_{T_0} & = 0.00764 s  \\
      U_{T_1} & = 0.00819 s  \\
      U_{m}   & = 0.0577 g   \\
    \end{aligned}
    \right.
  \end{equation}

  得到$G$的展伸不确定度为:

  \begin{equation}
    U_G/G = \sqrt{(U_L/L)^2 + 2 (U_{r_1}/r_1) ^2 +\ldots} = 0.00873
  \end{equation}

  \section{实验结论}

  \subsection{结果}
  \begin{equation}
    \left\{
    \begin{aligned}
      G & = 6.47 \pm 0.06 \times 10^{10} \\
      D & = 2.29 \pm 0.01 \times 10^{11} \\
      P & = 0.997                        \\
    \end{aligned}
    \right.
  \end{equation}

  \subsection{思考题}
  1. $\gamma = R \dfrac{\varphi}{L} = 1.392 \times 10 ^{-3} \lll 1$

  2. 实验中主要提升精度的办法是测多个周期取平均值,进而降低了误差的主要来源——人的反应时间;在具体测量时,更要仔细观察托盘是否水平,在转动扭摆时注意不能产生水平方向的摆动,以免影响测量精度。
\end{multicols}
\end{document}