% !TeX root = ./main.tex

\ustcsetup{
  title              = {中国科学技术大学\\学位论文模板示例文档 \ustcthesisversion},
  title*             = {An example of thesis template for University of Science
      and Technology of China \ustcthesisversion},
  author             = {李泽平},
  author*            = {Li Zeping},
  speciality         = {数学与应用数学},
  speciality*        = {Mathematics and Applied Mathematics},
  supervisor         = {XXX~教授, XXX~教授},
  supervisor*        = {Prof. XXX, Prof. XXX},
  % date               = {2017-05-01},  % 默认为今日
  % professional-type  = {专业学位类型},
  % professional-type* = {Professional degree type},
  % department         = {数学科学学院},  % 院系,本科生需要填写
  % student-id         = {PB11001000},  % 学号,本科生需要填写
  % secret-level       = {秘密},     % 绝密|机密|秘密|控阅,注释本行则公开
  % secret-level*      = {Secret},  % Top secret | Highly secret | Secret
  % secret-year        = {10},      % 保密/控阅期限
  % reviewer           = true,      % 声明页显示“评审专家签名”
  %
  % 数学字体
  % math-style         = GB,  % 可选:GB, TeX, ISO
  math-font          = xits,  % 可选:stix, xits, libertinus
}

\usepackage{amsthm}
\usepackage{graphicx}
\usepackage{booktabs}
\usepackage{longtable}
\usepackage[ruled,linesnumbered]{algorithm2e}
\usepackage{indentfirst}
\usepackage{zhnumber}
\usepackage{fancyhdr}
\usepackage{wrapfig}
\usepackage{caption}
\usepackage{float}
\usepackage{physics}
\usepackage[scientific-notation=true]{siunitx}

% 参考文献使用 BibLaTeX 宏包
\usepackage[style=ustcthesis-numeric]{biblatex}
% \usepackage[bibstyle=ustcthesis-numeric,citestyle=ustcthesis-inline,hyperref=true,backend=biber,sorting=none,backref=true]{biblatex}
% \usepackage[style=ustcthesis-authoryear]{biblatex}
% \usepackage[style=ustcthesis-bachelor]{biblatex}
% 声明 BibLaTeX 的数据库
\addbibresource{bib/ustc.bib}

% 配置图片的默认目录
\graphicspath{{figures/}}

% 数学命令
\makeatletter
\newcommand\dif{%  % 微分符号
  \mathop{}\!%
  \ifustc@math@style@TeX
    d%
  \else
    \mathrm{d}%
  \fi
}
\makeatother
\newcommand\eu{{\symup{e}}}
\newcommand\iu{{\symup{i}}}

% 用于写文档的命令
\DeclareRobustCommand\cs[1]{\texttt{\char`\\#1}}
\DeclareRobustCommand\pkg{\textsf}
\DeclareRobustCommand\file{\nolinkurl}

% hyperref 宏包在最后调用
\usepackage{hyperref}
