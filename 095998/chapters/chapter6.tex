% !TeX root = ../main.tex

\chapter{石墨烯的光学性质及其应用}

与传统半导体材料相比,石墨烯的热导率是硅的36倍以上,是砷化镓的100倍,载流子迁移率是硅的100被,是砷化镓的20倍,且不随温度变化。光学损伤阈值比砷化镓和硅高出了三个数量级。三阶光学非线性系数可以达到$10^{-7}$,比其他传统的半导体材料高出若干数量级。且光透率可以达到$97.7\%$,且有宽波段可调的光学性质。

除了以上特性之外,石墨烯的加工制备可以与现在半导体\textit{CMOS}工艺相兼容,加工、继承方便,具有独特的优势。

\section{石墨烯的光学性质}

参考图~\ref{fig:grapheneDiracPoint},对于本征石墨烯,其费米能点位于狄拉克点处,此时电子可以通过带间跃迁的方式从价带跃迁到能带。特别的,针对\textit{n}型或是\textit{p}型掺杂的石墨烯,费米能级会发生移动,以\textit{n}型掺杂为例,掺入的电子将填充导带底,致使费米能级上移,此时导带底部和价带顶部的电子吸收一定能量后都可发生跃迁。在如此特殊的能带结构下,石墨烯有着其他半导体材料所不具有的特殊的光学性质。

\subsection{线性光学性质}

由于石墨烯独特的电子能带结构,本征单层石墨烯的动力学光导与入射光频率无关,可用以下公式表示:

\begin{equation}
    G_1(\omega) = G_0 \equiv e^2/4\hbar
\end{equation}

其中,$\omega$为入射光频率,$e$为电子电荷,$\hbar$为普朗克常数。

本征石墨烯的光学透过率在宽光谱敢为内只取决于其精细结构常数$\alpha = e^2/\hbar c$($c$为光速),可用以下公式表示:

\begin{equation}
    T \equiv (1+2\pi G/c)^{-2} \approx 1-\pi \alpha
\end{equation}

另外,石墨烯带内光电导率和带间光导率均与化学式和入射光频率相关。值得注意的是,带内光电导率$\sigma_{intra}$与石墨烯的等离子增强效应和和表面等离基元传输密切相关。

\subsection{非线性光学性质}

当入射光所在电场与石墨烯内碳原子的外层电子发生共振时,石墨烯内电子云相对于原子核的位置发生偏移,产生极化,最终导致了石墨烯的非线性光学性质。

在外加光场的强度较弱时,产生的极化强度与外加电场$E$呈现线性依赖的关系,可用以下公式表示:

\begin{equation}
    P = \epsilon_0 x^{(1)}E
\end{equation}

其中$\epsilon_0$为真空介电常数,$x^{(1)}$为一阶线性极化率。

在外加光场的强度很强,电子云偏移较大的时候,电子极化强度$P$与$X$、$E$呈现非线性依赖关系:

\begin{equation}
    P = \epsilon_0 x^{(2)}E^2 + \epsilon_0 x^{(3)}E^3 + \ldots \epsilon_0 x^{(n)}E^n +\ldots
\end{equation}

其中$x^{(2)}$和$x^{(3)}$分别为二阶非线性极化率和三阶非线性极化率,均与石墨烯的饱和吸收特性、光学双稳态等非线性光学特性相关。

对于一节线性极化率$x^{(1)}$,实部代表了石墨烯折射率,虚部代表了光学增益或光学损耗。通过改变施加垂直表面的直流电场,可以有效调节$x^{(1)}$的数值,进而改变石墨烯的折射率。由于反演对称性,$x^{(2)}$通常认为为$0$。石墨烯的三阶非线性极化率$x^{(3)}$与石墨烯的许多非线性光学性质相关,如饱和吸收、自聚焦、克尔效应、光学双稳态以及孤波传播等。

\section{光学性质的应用}

光学性质的应用主要集中在以下方面:

\begin{itemize}
    \item 锁模光纤激光器
    \item 调 Q 光纤激光器
    \item 固体激光器
    \item 直波导结构光调制器
    \item 垂直透射式结构光调制器
    \item 超快、宽波段光探测器
    \item 共振腔增强的光探测器
    \item 波导型光探测器
    \item 叠层范德华异质结型光探测器
    \item 基于石墨烯等离子体的太赫兹激光器和天线
          % \item 发光二极管
\end{itemize}