% !TeX root = ../main.tex

\chapter{石墨烯的制备、表征}

\section{石墨烯的早期制备}

在石墨烯的早期发展上,率先被制备出来的是\textit{氧化石墨}和\textit{氧化石墨烯},伴随着后面本杰明·柯林斯·布罗迪对实验方法的改进,以及吕斯、沃格特、汉斯勃姆等人的研究,确认了氧化石墨烯在稀碱性介质中与肼、硫化氢或二价铁盐中会发生氧化还原反应,生成薄层状的碳。

但是早期制取石墨烯的过程并不特别成功。1999年,罗德尼·鲁夫带领的团队曾经尝试过利用硅片摩擦的办法制备石墨烯,虽然我们现在已经知道,这样的办法产物中能得到少层、甚至单层石墨烯,但鲁夫并没有对产物做进一步的研究,错过了发现石墨烯的机会。

荷兰物理学家赫勒采用了外延生长法的办法制备石墨烯,在2004年,他们的团队独立地利用碳化硅合成了石墨烯,并完成了单层石墨烯电学性质的测定,并发现了超薄外延石墨薄膜的二维电子气特性。但遗憾的是,赫勒并没有因此得到诺贝尔奖,但他的开拓性工作值得被认可。

在同一年,盖姆团队利用了一种简单的胶带分离的办法制备了近乎完美的石墨烯,这是第一次严格意义上单层原子厚度的石墨烯的发现,他们检测出石墨烯独特的电学性质,掀起了全世界研究石墨烯的热潮。

如同诺奖委员会所说的那样,“石墨烯研究的难点并不在于制备出石墨烯的结构,而是分离出足够大、单个的石墨烯来确认、表征、验证石墨烯的性质”,对于盖姆来说,利用透明胶带离析技术是重要一步,但重要一步是希望能够找到石墨烯令人惊奇的物理性质。最终在盖姆的学生诺沃肖洛夫的耐心下,他们发现石墨烯碎片有着高度导电的性质。

同年,他们在\textit{Science}上发表了一篇具有里程碑意义的论文:\textit{Electric Field Effect in Atomically Thin Carbon Films},这篇论文率先描述了单晶石墨薄膜能够在普通环境下稳定存在、并且具有高度导电的性质,打破了各种方法研究石墨烯单层都没有进展的僵局,推翻了几十年来二维晶体不能稳定存在的预言。